\documentclass[aspectratio=169, 10pt]{beamer}

\usetheme[progressbar=frametitle]{metropolis}
\usepackage{appendixnumberbeamer}

\usepackage{booktabs}
\usepackage[scale=2]{ccicons}

\usepackage{pgfplots}
\usepgfplotslibrary{dateplot}

\usepackage{xspace}
\newcommand{\themename}{\textbf{\textsc{metropolis}}\xspace}

\title{Evaluating Urban Simulations with a High Resolution Sensor Network in Baltimore, Maryland.}
\subtitle{}
\date{Dec 2017}
\author{Anna Scott, Hamada Badr, Darryn Waugh, Ben Zaitchik}
\institute{annascott@jhu.edu }


%%% commands to uncover 
\newcommand*{\vcenteredhbox}[1]{#1}

\begin{document}

\maketitle



\section{Introduction}

\begin{frame}
\frametitle{Literature Review: policy & reviews}
\begin{itemize}
\item UHI literature focuses on UHI attributing and understanding the causes \item UHI mitigation is less studied 
\item Cities are interested in UHI measures
\item Cties cite a lack of local data and knowledge as hindering their progress %\cite{hewitt2014cool}.
\item Cities list mitigating the impacts of heatwaves a key motivator of urban heat mitigation policies %\citep{hewitt2014cool}
\end{itemize}
\end{frame}

\begin{frame}
Based on this, I think these are important policy-relevant questions about the UHI:  
\begin{enumerate}
\item How does urban heat island intensity vary throughout the day? 
\item How variable is urban temperature? 
\item How does green infrastructure affect urban heat island intensity? 
\item How does urban heat intensity vary during heatwaves? 
\end{enumerate}
\end{frame}

\begin{frame}[label=intro]
\frametitle{Literature Review: WRF}
\centering
%\includegraphics[width=.5\textwidth]{figures/reported_deaths_by_climate_hazard.png}
Given that data is a limiting factor, models can be important. Looking at 50 most cited studies using  the WRF model \tiny{Chen, 2011}: 
\begin{itemize}
\item Urban simulations using conclude that WRF performs well for modelling the urban environment. 

\item Mean RMSEs for 2-meter temperature are $0.6-4.0^\circ C$.
 
\item Few studies evaluate WRF at the resolution at which it is run 

\item Most common evaluation metrics are statistics for single points like RMSE, correlation, bias, hit rate, index of agreement, and coefficient of determination
\end{itemize}

\end{frame}

\begin{frame}
\frametitle{Literature Review: WRF}
This literature doesn't evaluate WRF for: 
\begin{itemize}
\item geospatial variability
\item ability to evaluate policy measures
\end{itemize}
\end{frame}

\begin{frame}
\frametitle{Proposed Paper Outline}

Proposed paper outline: 
1) Using UHI observations, quantify: 
\begin{itemize}
\item Diurnal variation of urban heat island intensity
\item Spatial variability of urban temperature
\item Relationship between green infrastructure and urban heat island intensity
\item How urban heat intensity varies during heatwaves
\end{itemize}
2) Evaluate how well WRF reproduces these observed relationships 
\end{frame}
\section{Results}
\subsection{Observations}
\begin{frame}
\frametitle{Diurnal variation of urban heat intensity}
\end{frame}

\begin{frame}
\frametitle{Spatial variability of urban temperature}

\end{frame}

\begin{frame}
\frametitle{Green infrastructure and urban heat}

\end{frame}

\begin{frame}
\frametitle{Heatwaves}

\end{frame}

\subsection{Numerical Model}
\begin{frame}
\frametitle{Diurnal variation of urban heat intensity}
\end{frame}

\begin{frame}
\frametitle{Spatial variability of urban temperature}

\end{frame}

\begin{frame}
\frametitle{Green infrastructure and urban heat}

\end{frame}

\begin{frame}
\frametitle{Heatwaves}

\end{frame}

\section{Conclusions} 

\begin{frame}
\frametitle{Conclusion}
Observations from a low-cost sensor network and a 1km WRF run (UCM+ SSTs) show that : 
\begin{itemize}
\large{
\item Urban-rural nighttime differences smaller during hotter periods
\item Urban-rural nighttime differences smaller during moist weather types
\item During moist weather types, increased surface energy flux in rural areas
\item Dominant cause of changes in nighttime surface energy flux is net radiation 
}
\end{itemize}

\textbf{Questions?} annascott@jhu.edu / @annaunderthesun

\textsc{References}
\small{
\textbf{Scott, A.A.}, D.W. Waugh, and B. Zaitchik. Reduced Urband Heat Island Under Warmer Conditions, \textit{under review}, Environmental Research Letters, Oct 2017. 

Scott, Anna A., Ben Zaitchik, Darryn W. Waugh, and Katie O’Meara. "Intraurban Temperature Variability in Baltimore." Journal of Applied Meteorology and Climatology 56, no. 1 (2017): 159-171.
}
\end{frame}
%
%
%\begin{frame}[label=results]
%\frametitle{Results: JJA}
%\begin{figure}
%\centering
%\includegraphics[width=.8\textwidth]{../figures/JJAUHIBaltimin.png}
%\caption{Daily $T_r$ versus $\Delta T$ for JJA 1975-2015.}
%\end{figure}
% $\Delta T$ decreases when $T_r$ rises in Baltimore by $m_{JJA} = -0.26 ^\circ/$C/$^\circ$C. 
%%plot temperature vs. delta T for Baltimore 
%%Linear slope (least squares) $m$ is calculated for 58 cities. 
%\end{frame}
%
%\begin{frame}[label=30yrBalt]
%\frametitle{Results: mean JJA trends}
%\begin{figure}
%\centering 
%\includegraphics[width=\textwidth]{../figures/Baltimore.png}
%\caption{Mean JJA temperature (left) versus mean JJA $\Delta T$ for Baltimore.}
%\end{figure}
%We compare trends using $m_{30yr} = \frac{\beta_{\Delta T}}{\beta _{T_r}} $. Here, a warming Baltimore is associated with decreasing $\Delta T$. 
%
%\end{frame} 
 



%\begin{frame}[allowframebreaks]
%\frametitle{Bibliography}

%\alert{\footnotesize I am using the APA referencing/citation style in this presentation. \emph{You} should be using Harvard Cite-Them-Right style -- do not copy and paste from this list!}

%\nocite{*}
%\printbibliography[heading=none]

%\end{frame}

\end{document}